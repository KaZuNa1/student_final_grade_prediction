% Options for packages loaded elsewhere
% Options for packages loaded elsewhere
\PassOptionsToPackage{unicode}{hyperref}
\PassOptionsToPackage{hyphens}{url}
\PassOptionsToPackage{dvipsnames,svgnames,x11names}{xcolor}
%
\documentclass[
  mongolian,
  a4paperpaper,
]{article}
\usepackage{xcolor}
\usepackage[left=2cm,right=2cm,top=2.5cm,bottom=2.5cm]{geometry}
\usepackage{amsmath,amssymb}
\setcounter{secnumdepth}{5}
\usepackage{iftex}
\ifPDFTeX
  \usepackage[T1]{fontenc}
  \usepackage[utf8]{inputenc}
  \usepackage{textcomp} % provide euro and other symbols
\else % if luatex or xetex
  \usepackage{unicode-math} % this also loads fontspec
  \defaultfontfeatures{Scale=MatchLowercase}
  \defaultfontfeatures[\rmfamily]{Ligatures=TeX,Scale=1}
\fi
\usepackage{lmodern}
\ifPDFTeX\else
  % xetex/luatex font selection
  \setmainfont[]{Times New Roman}
  \setmonofont[]{Courier New}
\fi
% Use upquote if available, for straight quotes in verbatim environments
\IfFileExists{upquote.sty}{\usepackage{upquote}}{}
\IfFileExists{microtype.sty}{% use microtype if available
  \usepackage[]{microtype}
  \UseMicrotypeSet[protrusion]{basicmath} % disable protrusion for tt fonts
}{}
\makeatletter
\@ifundefined{KOMAClassName}{% if non-KOMA class
  \IfFileExists{parskip.sty}{%
    \usepackage{parskip}
  }{% else
    \setlength{\parindent}{0pt}
    \setlength{\parskip}{6pt plus 2pt minus 1pt}}
}{% if KOMA class
  \KOMAoptions{parskip=half}}
\makeatother
% Make \paragraph and \subparagraph free-standing
\makeatletter
\ifx\paragraph\undefined\else
  \let\oldparagraph\paragraph
  \renewcommand{\paragraph}{
    \@ifstar
      \xxxParagraphStar
      \xxxParagraphNoStar
  }
  \newcommand{\xxxParagraphStar}[1]{\oldparagraph*{#1}\mbox{}}
  \newcommand{\xxxParagraphNoStar}[1]{\oldparagraph{#1}\mbox{}}
\fi
\ifx\subparagraph\undefined\else
  \let\oldsubparagraph\subparagraph
  \renewcommand{\subparagraph}{
    \@ifstar
      \xxxSubParagraphStar
      \xxxSubParagraphNoStar
  }
  \newcommand{\xxxSubParagraphStar}[1]{\oldsubparagraph*{#1}\mbox{}}
  \newcommand{\xxxSubParagraphNoStar}[1]{\oldsubparagraph{#1}\mbox{}}
\fi
\makeatother

\usepackage{color}
\usepackage{fancyvrb}
\newcommand{\VerbBar}{|}
\newcommand{\VERB}{\Verb[commandchars=\\\{\}]}
\DefineVerbatimEnvironment{Highlighting}{Verbatim}{commandchars=\\\{\}}
% Add ',fontsize=\small' for more characters per line
\usepackage{framed}
\definecolor{shadecolor}{RGB}{241,243,245}
\newenvironment{Shaded}{\begin{snugshade}}{\end{snugshade}}
\newcommand{\AlertTok}[1]{\textcolor[rgb]{0.68,0.00,0.00}{#1}}
\newcommand{\AnnotationTok}[1]{\textcolor[rgb]{0.37,0.37,0.37}{#1}}
\newcommand{\AttributeTok}[1]{\textcolor[rgb]{0.40,0.45,0.13}{#1}}
\newcommand{\BaseNTok}[1]{\textcolor[rgb]{0.68,0.00,0.00}{#1}}
\newcommand{\BuiltInTok}[1]{\textcolor[rgb]{0.00,0.23,0.31}{#1}}
\newcommand{\CharTok}[1]{\textcolor[rgb]{0.13,0.47,0.30}{#1}}
\newcommand{\CommentTok}[1]{\textcolor[rgb]{0.37,0.37,0.37}{#1}}
\newcommand{\CommentVarTok}[1]{\textcolor[rgb]{0.37,0.37,0.37}{\textit{#1}}}
\newcommand{\ConstantTok}[1]{\textcolor[rgb]{0.56,0.35,0.01}{#1}}
\newcommand{\ControlFlowTok}[1]{\textcolor[rgb]{0.00,0.23,0.31}{\textbf{#1}}}
\newcommand{\DataTypeTok}[1]{\textcolor[rgb]{0.68,0.00,0.00}{#1}}
\newcommand{\DecValTok}[1]{\textcolor[rgb]{0.68,0.00,0.00}{#1}}
\newcommand{\DocumentationTok}[1]{\textcolor[rgb]{0.37,0.37,0.37}{\textit{#1}}}
\newcommand{\ErrorTok}[1]{\textcolor[rgb]{0.68,0.00,0.00}{#1}}
\newcommand{\ExtensionTok}[1]{\textcolor[rgb]{0.00,0.23,0.31}{#1}}
\newcommand{\FloatTok}[1]{\textcolor[rgb]{0.68,0.00,0.00}{#1}}
\newcommand{\FunctionTok}[1]{\textcolor[rgb]{0.28,0.35,0.67}{#1}}
\newcommand{\ImportTok}[1]{\textcolor[rgb]{0.00,0.46,0.62}{#1}}
\newcommand{\InformationTok}[1]{\textcolor[rgb]{0.37,0.37,0.37}{#1}}
\newcommand{\KeywordTok}[1]{\textcolor[rgb]{0.00,0.23,0.31}{\textbf{#1}}}
\newcommand{\NormalTok}[1]{\textcolor[rgb]{0.00,0.23,0.31}{#1}}
\newcommand{\OperatorTok}[1]{\textcolor[rgb]{0.37,0.37,0.37}{#1}}
\newcommand{\OtherTok}[1]{\textcolor[rgb]{0.00,0.23,0.31}{#1}}
\newcommand{\PreprocessorTok}[1]{\textcolor[rgb]{0.68,0.00,0.00}{#1}}
\newcommand{\RegionMarkerTok}[1]{\textcolor[rgb]{0.00,0.23,0.31}{#1}}
\newcommand{\SpecialCharTok}[1]{\textcolor[rgb]{0.37,0.37,0.37}{#1}}
\newcommand{\SpecialStringTok}[1]{\textcolor[rgb]{0.13,0.47,0.30}{#1}}
\newcommand{\StringTok}[1]{\textcolor[rgb]{0.13,0.47,0.30}{#1}}
\newcommand{\VariableTok}[1]{\textcolor[rgb]{0.07,0.07,0.07}{#1}}
\newcommand{\VerbatimStringTok}[1]{\textcolor[rgb]{0.13,0.47,0.30}{#1}}
\newcommand{\WarningTok}[1]{\textcolor[rgb]{0.37,0.37,0.37}{\textit{#1}}}

\usepackage{longtable,booktabs,array}
\usepackage{calc} % for calculating minipage widths
% Correct order of tables after \paragraph or \subparagraph
\usepackage{etoolbox}
\makeatletter
\patchcmd\longtable{\par}{\if@noskipsec\mbox{}\fi\par}{}{}
\makeatother
% Allow footnotes in longtable head/foot
\IfFileExists{footnotehyper.sty}{\usepackage{footnotehyper}}{\usepackage{footnote}}
\makesavenoteenv{longtable}
\usepackage{graphicx}
\makeatletter
\newsavebox\pandoc@box
\newcommand*\pandocbounded[1]{% scales image to fit in text height/width
  \sbox\pandoc@box{#1}%
  \Gscale@div\@tempa{\textheight}{\dimexpr\ht\pandoc@box+\dp\pandoc@box\relax}%
  \Gscale@div\@tempb{\linewidth}{\wd\pandoc@box}%
  \ifdim\@tempb\p@<\@tempa\p@\let\@tempa\@tempb\fi% select the smaller of both
  \ifdim\@tempa\p@<\p@\scalebox{\@tempa}{\usebox\pandoc@box}%
  \else\usebox{\pandoc@box}%
  \fi%
}
% Set default figure placement to htbp
\def\fps@figure{htbp}
\makeatother



\ifLuaTeX
\usepackage[bidi=basic,provide=*]{babel}
\else
\usepackage[bidi=default,provide=*]{babel}
\fi
\ifPDFTeX
\else
\babelfont{rm}[]{Times New Roman}
\fi
% get rid of language-specific shorthands (see #6817):
\let\LanguageShortHands\languageshorthands
\def\languageshorthands#1{}


\setlength{\emergencystretch}{3em} % prevent overfull lines

\providecommand{\tightlist}{%
  \setlength{\itemsep}{0pt}\setlength{\parskip}{0pt}}



 


\usepackage{polyglossia}
\setmainlanguage{mongolian}
\newfontfamily\mongoliafont[Script=Cyrillic]{Times New Roman}
\newfontfamily\cyrillicfonttt{Courier New}
\newfontfamily\cyrillicfontsf{Arial}
\makeatletter
\@ifpackageloaded{caption}{}{\usepackage{caption}}
\AtBeginDocument{%
\ifdefined\contentsname
  \renewcommand*\contentsname{Table of contents}
\else
  \newcommand\contentsname{Table of contents}
\fi
\ifdefined\listfigurename
  \renewcommand*\listfigurename{List of Figures}
\else
  \newcommand\listfigurename{List of Figures}
\fi
\ifdefined\listtablename
  \renewcommand*\listtablename{List of Tables}
\else
  \newcommand\listtablename{List of Tables}
\fi
\ifdefined\figurename
  \renewcommand*\figurename{Figure}
\else
  \newcommand\figurename{Figure}
\fi
\ifdefined\tablename
  \renewcommand*\tablename{Table}
\else
  \newcommand\tablename{Table}
\fi
}
\@ifpackageloaded{float}{}{\usepackage{float}}
\floatstyle{ruled}
\@ifundefined{c@chapter}{\newfloat{codelisting}{h}{lop}}{\newfloat{codelisting}{h}{lop}[chapter]}
\floatname{codelisting}{Listing}
\newcommand*\listoflistings{\listof{codelisting}{List of Listings}}
\makeatother
\makeatletter
\makeatother
\makeatletter
\@ifpackageloaded{caption}{}{\usepackage{caption}}
\@ifpackageloaded{subcaption}{}{\usepackage{subcaption}}
\makeatother
\usepackage{bookmark}
\IfFileExists{xurl.sty}{\usepackage{xurl}}{} % add URL line breaks if available
\urlstyle{same}
\hypersetup{
  pdftitle={Оюутны эцсийн дүнг урьдчилан таамаглах},
  pdfauthor={Багийн гишүүдийн нэрс},
  pdflang={mn},
  colorlinks=true,
  linkcolor={blue},
  filecolor={Maroon},
  citecolor={Blue},
  urlcolor={Blue},
  pdfcreator={LaTeX via pandoc}}


\title{Оюутны эцсийн дүнг урьдчилан таамаглах}
\author{Багийн гишүүдийн нэрс}
\date{Invalid Date}
\begin{document}
\maketitle

\renewcommand*\contentsname{Агуулга}
{
\hypersetup{linkcolor=}
\setcounter{tocdepth}{3}
\tableofcontents
}

Энэхүү төслийн ажлын зорилго нь Португалийн дунд сургуулийн математикийн
хичээлийн оюутнуудын эцсийн дүн (G3)-ийг урьдчилан таамаглахад шугаман
регрессийн загварыг ашиглан статистикийн арга зүйг практикт хэрэглэх юм.
Өгөгдлийн олонлогт 395 оюутны 33 шинж чанарын мэдээлэл багтсан бөгөөд
загварын үр дүнг үнэлсэн. Үр дүнгээр оюутнуудын хоёр дахь үеийн дүн (G2)
нь эцсийн дүнг урьдчилан таамаглахад хамгийн чухал хүчин зүйл бөгөөд
загварын R² үзүүлэлт 0.75 байв.

\section{Оршил}\label{ux43eux440ux448ux438ux43b}

{[}Яагаад оюутны дүнг урьдчилан таамаглах чухал вэ? Машин сургалт
боловсролд хэрхэн тусалдаг талаар{]}

\subsection{Төслийн ажлын
зорилго}\label{ux442ux4e9ux441ux43bux438ux439ux43d-ux430ux436ux43bux44bux43d-ux437ux43eux440ux438ux43bux433ux43e}

Төслийн ажлын үндсэн зорилгууд:

\begin{enumerate}
\def\labelenumi{\arabic{enumi}.}
\tightlist
\item
  {[}Зорилго 1{]}
\item
  {[}Зорилго 2{]}
\item
  {[}Зорилго 3{]}
\item
  {[}Зорилго 4{]}
\end{enumerate}

\section{Өгөгдлийн эх сурвалж ба
тайлбар}\label{ux4e9ux433ux4e9ux433ux434ux43bux438ux439ux43d-ux44dux445-ux441ux443ux440ux432ux430ux43bux436-ux431ux430-ux442ux430ux439ux43bux431ux430ux440}

\subsection{Эх
сурвалж}\label{ux44dux445-ux441ux443ux440ux432ux430ux43bux436}

{[}Kaggle холбоос, Португалийн оюутнууд, математикийн хичээл{]}

\subsection{Өгөгдлийн
тайлбар}\label{ux4e9ux433ux4e9ux433ux434ux43bux438ux439ux43d-ux442ux430ux439ux43bux431ux430ux440}

{[}395 оюутан, 33 шинж чанар{]}

\textbf{Демографик мэдээлэл:}

{[}school, sex, age, address, famsize, Pstatus{]}

\textbf{Боловсролын мэдээлэл:}

{[}Medu, Fedu, studytime, failures, schoolsup, absences{]}

\textbf{Сурлагын дүн:}

{[}G1, G2, G3 тайлбар{]}

\subsection{Өгөгдлийг уншиж
танилцах}\label{ux4e9ux433ux4e9ux433ux434ux43bux438ux439ux433-ux443ux43dux448ux438ux436-ux442ux430ux43dux438ux43bux446ux430ux445}

\begin{Shaded}
\begin{Highlighting}[]
\CommentTok{\# Сангуудыг импортлох}
\CommentTok{\# pandas, numpy, matplotlib, seaborn}
\CommentTok{\# Өгөгдлийг уншиж, хэмжээг харуулах}
\end{Highlighting}
\end{Shaded}

{[}Өгөгдөл амжилттай ачаалагдсан тухай{]}

\begin{Shaded}
\begin{Highlighting}[]
\CommentTok{\# df.head() {-} эхний 5 мөр}
\end{Highlighting}
\end{Shaded}

\begin{Shaded}
\begin{Highlighting}[]
\CommentTok{\# df.describe() {-} статистик үзүүлэлтүүд}
\end{Highlighting}
\end{Shaded}

\begin{Shaded}
\begin{Highlighting}[]
\CommentTok{\# Дутуу утга шалгах}
\end{Highlighting}
\end{Shaded}

{[}Дутуу утга байхгүй{]}

\section{Хэрэглэсэн арга загварын
танилцуулга}\label{ux445ux44dux440ux44dux433ux43bux44dux441ux44dux43d-ux430ux440ux433ux430-ux437ux430ux433ux432ux430ux440ux44bux43d-ux442ux430ux43dux438ux43bux446ux443ux443ux43bux433ux430}

\subsection{Шугаман
регресс}\label{ux448ux443ux433ux430ux43cux430ux43d-ux440ux435ux433ux440ux435ux441ux441}

{[}Шугаман регресс юу вэ, яагаад хэрэглэх вэ{]}

\subsection{Математик
загвар}\label{ux43cux430ux442ux435ux43cux430ux442ux438ux43a-ux437ux430ux433ux432ux430ux440}

Шугаман регрессийн ерөнхий загвар:

\[y = \beta_0 + \beta_1x_1 + \beta_2x_2 + ... + \beta_nx_n + \varepsilon\]

Энд:

\begin{itemize}
\tightlist
\item
  {[}y тайлбар{]}
\item
  {[}β₀ тайлбар{]}
\item
  {[}βᵢ тайлбар{]}
\item
  {[}xᵢ тайлбар{]}
\item
  {[}ε тайлбар{]}
\end{itemize}

\subsection{Загварын давуу
тал}\label{ux437ux430ux433ux432ux430ux440ux44bux43d-ux434ux430ux432ux443ux443-ux442ux430ux43b}

{[}Коэффициент ойлгогдох, хэрэгжүүлэхэд хялбар, хурдан, статистик
үндэслэлтэй гэх мэт{]}

\section{Өгөгдөлтэй танилцах
шинжилгээ}\label{ux4e9ux433ux4e9ux433ux434ux4e9ux43bux442ux44dux439-ux442ux430ux43dux438ux43bux446ux430ux445-ux448ux438ux43dux436ux438ux43bux433ux44dux44d}

\subsection{Зорилтот хувьсагчийн шинжилгээ
(G3)}\label{ux437ux43eux440ux438ux43bux442ux43eux442-ux445ux443ux432ux44cux441ux430ux433ux447ux438ux439ux43d-ux448ux438ux43dux436ux438ux43bux433ux44dux44d-g3}

\begin{Shaded}
\begin{Highlighting}[]
\CommentTok{\# G3 статистик (дундаж, медиан, std, min, max)}
\CommentTok{\# Histogram + boxplot зурах}
\end{Highlighting}
\end{Shaded}

{[}G3 тархалтын тайлбар{]}

\subsection{Корреляцийн
шинжилгээ}\label{ux43aux43eux440ux440ux435ux43bux44fux446ux438ux439ux43d-ux448ux438ux43dux436ux438ux43bux433ux44dux44d}

\begin{Shaded}
\begin{Highlighting}[]
\CommentTok{\# G1, G2, G3 корреляцийн матриц}
\CommentTok{\# Heatmap зурах}
\end{Highlighting}
\end{Shaded}

{[}G2-G3 корреляци өндөр тухай{]}

\begin{Shaded}
\begin{Highlighting}[]
\CommentTok{\# G1 vs G3, G2 vs G3 scatter plots}
\end{Highlighting}
\end{Shaded}

{[}Шугаман хамаарал харагдаж байна{]}

\subsection{Категори хувьсагчдын
шинжилгээ}\label{ux43aux430ux442ux435ux433ux43eux440ux438-ux445ux443ux432ux44cux441ux430ux433ux447ux434ux44bux43d-ux448ux438ux43dux436ux438ux43bux433ux44dux44d}

\begin{Shaded}
\begin{Highlighting}[]
\CommentTok{\# sex, school, internet, higher гэх мэтийн boxplot{-}ууд}
\end{Highlighting}
\end{Shaded}

{[}Дээд боловсрол хүсч буй оюутнууд илүү өндөр дүнтэй{]}

\subsection{Тоон хувьсагчдын
корреляци}\label{ux442ux43eux43eux43d-ux445ux443ux432ux44cux441ux430ux433ux447ux434ux44bux43d-ux43aux43eux440ux440ux435ux43bux44fux446ux438}

\begin{Shaded}
\begin{Highlighting}[]
\CommentTok{\# Тоон хувьсагчдын G3{-}тай корреляци}
\CommentTok{\# Bar chart}
\end{Highlighting}
\end{Shaded}

{[}failures хамгийн сөрөг корреляцитай{]}

\section{Загварыг хэрэгжүүлсэн
алхмууд}\label{ux437ux430ux433ux432ux430ux440ux44bux433-ux445ux44dux440ux44dux433ux436ux4afux4afux43bux441ux44dux43d-ux430ux43bux445ux43cux443ux443ux434}

\subsection{Шаардлагатай сангуудыг
импортлох}\label{ux448ux430ux430ux440ux434ux43bux430ux433ux430ux442ux430ux439-ux441ux430ux43dux433ux443ux443ux434ux44bux433-ux438ux43cux43fux43eux440ux442ux43bux43eux445}

\begin{Shaded}
\begin{Highlighting}[]
\CommentTok{\# sklearn импортлох}
\CommentTok{\# train\_test\_split, LinearRegression, LabelEncoder}
\CommentTok{\# mean\_squared\_error, r2\_score гэх мэт}
\end{Highlighting}
\end{Shaded}

{[}Эдгээр сангууд юунд ашиглагдах{]}

\subsection{Өгөгдөл
боловсруулалт}\label{ux4e9ux433ux4e9ux433ux434ux4e9ux43b-ux431ux43eux43bux43eux432ux441ux440ux443ux443ux43bux430ux43bux442}

\begin{Shaded}
\begin{Highlighting}[]
\CommentTok{\# Категори хувьсагчдыг кодлох (Label Encoding)}
\end{Highlighting}
\end{Shaded}

{[}Текст утгуудыг тоонд хөрвүүлсэн{]}

\begin{Shaded}
\begin{Highlighting}[]
\CommentTok{\# X (features) болон y (target) ялгах}
\end{Highlighting}
\end{Shaded}

\subsection{Сургалт ба тестийн
олонлог}\label{ux441ux443ux440ux433ux430ux43bux442-ux431ux430-ux442ux435ux441ux442ux438ux439ux43d-ux43eux43bux43eux43dux43bux43eux433}

\begin{Shaded}
\begin{Highlighting}[]
\CommentTok{\# train\_test\_split 80/20}
\end{Highlighting}
\end{Shaded}

{[}316 сургалт, 79 тест{]}

\subsection{Загварыг
сургах}\label{ux437ux430ux433ux432ux430ux440ux44bux433-ux441ux443ux440ux433ux430ux445}

\begin{Shaded}
\begin{Highlighting}[]
\CommentTok{\# LinearRegression()}
\CommentTok{\# model.fit()}
\end{Highlighting}
\end{Shaded}

\subsection{Шинж чанаруудын
коэффициентүүд}\label{ux448ux438ux43dux436-ux447ux430ux43dux430ux440ux443ux443ux434ux44bux43d-ux43aux43eux44dux444ux444ux438ux446ux438ux435ux43dux442ux4afux4afux434}

\begin{Shaded}
\begin{Highlighting}[]
\CommentTok{\# Коэффициентүүдийг dataframe{-}д оруулах}
\CommentTok{\# Bar chart зурах}
\CommentTok{\# Топ 10 харуулах}
\end{Highlighting}
\end{Shaded}

{[}G2 хамгийн өндөр коэффициенттэй (0.95){]}

\section{Үр дүн ба загварын
үнэлгээ}\label{ux4afux440-ux434ux4afux43d-ux431ux430-ux437ux430ux433ux432ux430ux440ux44bux43d-ux4afux43dux44dux43bux433ux44dux44d}

\subsection{Таамаглал}\label{ux442ux430ux430ux43cux430ux433ux43bux430ux43b}

\begin{Shaded}
\begin{Highlighting}[]
\CommentTok{\# y\_pred = model.predict()}
\CommentTok{\# Comparison table (бодит vs таамаглал)}
\end{Highlighting}
\end{Shaded}

\subsection{Загварын
гүйцэтгэл}\label{ux437ux430ux433ux432ux430ux440ux44bux43d-ux433ux4afux439ux446ux44dux442ux433ux44dux43b}

\begin{Shaded}
\begin{Highlighting}[]
\CommentTok{\# MSE, RMSE, MAE, R² тооцоолох}
\CommentTok{\# Үр дүнг хэвлэх}
\end{Highlighting}
\end{Shaded}

{[}R² = 0.75, MAE = 1.5 тайлбар{]}

\begin{Shaded}
\begin{Highlighting}[]
\CommentTok{\# Бодит vs таамаглал scatter plot}
\end{Highlighting}
\end{Shaded}

{[}Улаан шугамд ойрхон = сайн таамаглал{]}

\subsection{Үлдэгдлийн
шинжилгээ}\label{ux4afux43bux434ux44dux433ux434ux43bux438ux439ux43d-ux448ux438ux43dux436ux438ux43bux433ux44dux44d}

\begin{Shaded}
\begin{Highlighting}[]
\CommentTok{\# Residual scatter plot}
\CommentTok{\# Residual histogram}
\end{Highlighting}
\end{Shaded}

{[}Үлдэгдэл нормал тархалттай = таамаглал хангагдсан{]}

\subsection{Хөндлөн
баталгаажуулалт}\label{ux445ux4e9ux43dux434ux43bux4e9ux43d-ux431ux430ux442ux430ux43bux433ux430ux430ux436ux443ux443ux43bux430ux43bux442}

\begin{Shaded}
\begin{Highlighting}[]
\CommentTok{\# 5{-}fold cross{-}validation}
\CommentTok{\# R² оноонуудыг харуулах}
\end{Highlighting}
\end{Shaded}

{[}Дундаж R² = 0.79, тогтвортой{]}

\section{Дүгнэлт}\label{ux434ux4afux433ux43dux44dux43bux442}

\subsection{Үндсэн
дүгнэлтүүд}\label{ux4afux43dux434ux441ux44dux43d-ux434ux4afux433ux43dux44dux43bux442ux4afux4afux434}

Төслийн ажлын үндсэн дүгнэлтүүд:

\begin{enumerate}
\def\labelenumi{\arabic{enumi}.}
\item
  \textbf{{[}G2-ийн тухай дүгнэлт{]}}
\item
  \textbf{{[}Загварын нарийвчлалын тухай дүгнэлт{]}}
\item
  \textbf{{[}Failures-ийн сөрөг нөлөөний тухай{]}}
\item
  \textbf{{[}Бусад хүчин зүйлсийн тухай{]}}
\item
  \textbf{{[}Cross-validation-ий тухай{]}}
\end{enumerate}

\subsection{Практик
хэрэглээ}\label{ux43fux440ux430ux43aux442ux438ux43a-ux445ux44dux440ux44dux433ux43bux44dux44d}

{[}Хэрхэн хэрэглэж болох - эрт таних, багш дэмжлэг гэх мэт{]}

\subsection{Хязгаарлалтууд}\label{ux445ux44fux437ux433ux430ux430ux440ux43bux430ux43bux442ux443ux443ux434}

{[}Зөвхөн Португалийн өгөгдөл, корреляци≠шалтгаан, G2 шаардлагатай{]}

\subsection{Цаашдын
судалгаа}\label{ux446ux430ux430ux448ux434ux44bux43d-ux441ux443ux434ux430ux43bux433ux430ux430}

{[}Бусад загвар туршиж үзэх, feature engineering, илүү их өгөгдөл{]}

\section{Багийн гишүүдийн үүрэг
оролцоо}\label{ux431ux430ux433ux438ux439ux43d-ux433ux438ux448ux4afux4afux434ux438ux439ux43d-ux4afux4afux440ux44dux433-ux43eux440ux43eux43bux446ux43eux43e}

\begin{longtable}[]{@{}lll@{}}
\toprule\noalign{}
Гишүүний нэр & Үүрэг & Хувь нэмэр \\
\midrule\noalign{}
\endhead
\bottomrule\noalign{}
\endlastfoot
{[}Нэр 1{]} & {[}Үүрэг 1{]} & 25\% \\
{[}Нэр 2{]} & {[}Үүрэг 2{]} & 30\% \\
{[}Нэр 3{]} & {[}Үүрэг 3{]} & 25\% \\
{[}Нэр 4{]} & {[}Үүрэг 4{]} & 20\% \\
\end{longtable}

\textbf{Тэмдэглэл:} {[}Бүх гишүүд идэвхтэй оролцсон{]}

\section*{Ашигласан
материал}\label{ux430ux448ux438ux433ux43bux430ux441ux430ux43d-ux43cux430ux442ux435ux440ux438ux430ux43b}
\addcontentsline{toc}{section}{Ашигласан материал}

\phantomsection\label{refs}




\end{document}
